\documentclass{article}

% \prooftree and related commands
\usepackage{ebproof}
% most symbols
\usepackage{amsmath}
\usepackage{amssymb}
% \RHD
\usepackage[nointegrals]{wasysym}
\usepackage{xcolor}

\newcommand{\ignore}[1]{}
\newcommand{\todo}[1]{\colorbox{red}{#1}}
\newcommand{\consider}[1]{\colorbox{red}{!!!} #1 \colorbox{red}{!!!}}
\newcommand{\code}[1]{\texttt{#1}}

\renewcommand{\implies}{\Rightarrow}
\renewcommand{\impliedby}{\Leftarrow}
\newcommand{\rcd}[1]{\{#1\}}
\newcommand{\B}{\mathcal{B}}
\newcommand{\C}{\mathcal{C}}
\newcommand{\D}{\mathcal{D}}
\newcommand{\rowvar}{\alpha_\rho}
\newcommand{\bottom}{\perp}
\newcommand{\define}{::=}
\newcommand{\marker}[1]{\RHD_{#1}}
\newcommand{\subsume}{<:}
\newcommand{\synthesizes}{\Rightarrow \!\!\! \Rightarrow}
\newcommand{\app}{\bullet}
\newcommand{\prj}{\,\pmb{\#}\,}
\newcommand{\instLSymbol}{\;\substack{<\\:=}\;}
\newcommand{\instRSymbol}{\;\substack{<\\=:}\;}
\newcommand{\ev}{\hat}
\newcommand{\spc}{\qquad}
\newcommand{\apply}[1]{\left[#1\right]}
\newcommand{\eva}[1][]{\ev \alpha_{#1}}
\newcommand{\evb}{\ev \beta}

\newcommand{\rnil}{\{\}}
\newcommand{\rcons}[2]{\{#1 \,|\, #2\}}

\newcommand{\wf}[2]{#1 \vdash #2}

\newcommand{\subtype}{\le}
\newcommand{\subtypes}[3]{#1 \vdash #2 \le #3}

\newcommand{\synth}[4]{#1 \vdash #2 \Rightarrow #3 \dashv #4}
\renewcommand{\check}[4]{#1 \vdash #2 \Leftarrow #3 \dashv #4}

\newcommand{\presynth}[6]{#1 \vdash #2 #3 #4 \synthesizes #5 \dashv #6}

\newcommand{\subsumes}[4]{#1 \vdash #2 \subsume #3 \dashv #4}

\newcommand{\instL}[4]{#1 \vdash #2 \instLSymbol #3 \dashv #4}
\newcommand{\instR}[4]{#1 \vdash #2 \instRSymbol #3 \dashv #4}

\newcommand{\lookup}[5]{#1 \vdash #2 \# #3 \longrightarrow #4 \dashv #5}

\newcommand{\deduct}[3][]
{
  \begin{prooftree}
    \hypo{#2}
    \infer1[\text{#1}]{#3}
  \end{prooftree}
}

\begin{document}

\section{Introduction}
\consider{These rules are WIP - the ones that come directly from
  \textit{Complete and Easy} should be right, but the rest have not been proven}

We present a type inference and checking system based off of the one in
\textit{Complete and Easy Bidirectional Typechecking for Higher-Rank
  Polymorphism} by Dunfield and Krishnaswami. This system is extended to
encompass record types.

Future work may include extending to full(er) subtyping, as well as variant
types.

\section{Grammar}

\subsection{Types}
\[\begin{array}{rcl}
\B & \define & \texttt{Num} \mid \texttt{Bool} \mid \cdots
\\
% c & \define & \bottom \mid \tau \mid \top
% \\
A & \define & \B \mid \ev\alpha \mid \alpha \mid \forall \alpha. A \mid A_0 \to A_1 \mid \{R\}
\\
\tau & \define & \B \mid \ev\alpha \mid \alpha \mid \tau_0 \to \tau_1 \mid \{\rho\}
\\
R & \define & \cdot \mid \ev\alpha \mid \ell : A, R
\\
r & \define & \cdot \mid \ell : A, r
\\
\rho & \define & \cdot \mid \ev\alpha \mid \ell : \tau, \rho
\end{array}
\]

\subsection{Terms}
TBD

\subsection{Records}
TBD

\section{Conventions}

We sometimes use $\beta$ and $\gamma$ for type variables in addition
to $\alpha$. We adopt the same
conventions Dunfield and Krishnaswami use for their typing rules.

In record types, we elide the final empty row type
($\cdot$); e.g. we write $\{\ell_0 : A_0, \ell_1 : A_1\}$ instead of $\{\ell_0 : A_0, \ell_1 : A_1, \cdot\}$.

% We treat rows sort of like sets, even though we represent them for simplicity's
% sake as cons lists in the Grammar. Whenever we have syntax that looks like
% \[\subsumes{\Gamma}{\ell_1 : A_1, \ell_2 : A_2, \dots}{\ell_1 : B_1, \ell_2 :
%     B_2, \dots, m_1 : B_1', m_2 : B_2', \dots}{\Delta}\]
% We are implicitly reordering the sets so that the labels that match are at the
% front and the mismatched labels are at the back. This is important for the row
% subsumption rules, since it means that we can move the mismatched label
% resolution to a base case. An implementation of these rules would presumably use
% something easier, like set difference, to figure out which labels to compare.

\section{Typing Rules}
We define algorithmic typing with the following judgments:
\[
\check{\Gamma}{e}{A}{\Delta}
\spc
\synth{\Gamma}{e}{A}{\Delta}
\]
which respectively represent type checking (inputs: $\Gamma$, $e$, $A$; output: $\Delta$) and type synthesis (inputs: $\Gamma$, $e$; outputs: $A$, $\Delta$).

We also define a binary algorithmic judgement:

\[
\presynth{\Gamma}{X}{\square}{Y}{Z}{\Delta}
\]

which represents a binary judgement \(\square\) under the context \(\Gamma\) on
values \(X\) and \(Y\) that synthesizes \(Z\) with output context \(\Delta\).
For example, the syntax that Dunfield and Krishnaswami use for function
application synthesis judgements would be

\[
\presynth \Gamma A \app e C \Delta
\]

which means that under context \(\Gamma\), \(A\) applied to the term \(e\)
synthesizes output type \(C\) and context \(\Delta\).

\[
  \deduct[(Var)]
  {
    (x : A) \in \Gamma
  }
  { \synth{\Gamma}{x}{A}{\Gamma} }
  \spc
  \deduct[(Sub)]
  {
    \synth{\Gamma}{e}{A}{\Theta} \spc
    \subsumes{\Theta}{\apply\Theta A}{\apply\Theta B}{\Delta}
  }
  { \check{\Gamma}{e}{B}{\Delta} }
\]

\[
  \deduct[(Annotation)]
  { \Gamma \vdash A \spc \check{\Gamma}{e}{A}{\Delta} }
  { \synth{\Gamma}{(e : A)}{A}{\Delta} }
  \spc
  \deduct[(\(\forall\) I)]
  { \check{\Gamma, \alpha}{e}{A}{\Delta, \alpha, \Theta} }
  { \check{\Gamma}{e}{\forall \alpha. A}{\Delta} }
\]

\[
  \deduct[(\(\to\)I)]
  { \check{\Gamma, x : A}{e}{B}{\Delta, x : A, \Theta} }
  { \check{\Gamma}{\lambda x. e}{A \to B}{\Delta} }
  \spc
  \deduct[(\(\to\)I\(\implies\))]
  { \check{\Gamma, \ev\alpha, \ev\beta, x : \ev\alpha}{e}{\ev\beta}{\Delta, x : \ev\alpha, \Theta} }
  { \synth{\Gamma}{\lambda x. e}{\ev\alpha \to \ev\beta}{\Delta} }
\]

\[
  \deduct[(\(\to\)E)]
  { \synth{\Gamma}{e_1}{A}{\Theta} \spc \presynth{\Theta}{\apply\Theta A}{\app}{e_2}{C}{\Delta} }
  { \synth{\Gamma}{e_1 \ e_2}{C}{\Delta}  }
  \spc
  \deduct[(\(\forall\) App)]
  { \presynth{\Gamma, \ev\alpha}{A[\alpha := \ev\alpha]}{\app}{e}{C}{\Delta} }
  { \presynth{\Gamma}{\forall \alpha. A}{\app}{e}{C}{\Delta} }
\]

\[
  \deduct[(\(\to\)App)]
    { \check{\Gamma}{e}{A}{\Delta} }
    { \presynth{\Gamma}{A \to C}{\app}{e}{C}{\Delta} }
\]

\[
  \deduct[(\(\ev\alpha\)App)]
    {
      \check{\Gamma[\ev{\alpha_2}, \ev{\alpha_1}, \ev \alpha = \ev{\alpha_1} \to
        \ev{\alpha_2}]}{e}{\ev{\alpha_1}}{\Delta}
    }
    { \presynth{\Gamma[\ev\alpha]}{\ev\alpha}{\app}{e}{\ev\alpha_2}{\Delta} }
\]


\[
\deduct[(\{\}I)]{}{\check{\Gamma}{\{\}}{\{\}}{\Gamma}}
\spc
\deduct[(\{\}I\(\implies\))]{}{\synth{\Gamma}{\{\}}{\{\}}{\Gamma}}
\]

\[
  \deduct[(RcdI)]
  { \check{\Gamma}{e}{A}{\Theta} \spc \check{\Theta}{\apply\Theta r}{\apply\Theta\rho}{\Delta} }
  { \check{\Gamma}{\{\ell : e , r \}}{\{\ell : A, \rho\}}{\Delta} }
  \spc
  \deduct[(RcdI\(\implies\))]
  { \synth{\Gamma}{e}{A}{\Theta} \spc \synth{\Theta}{\apply\Theta r}{\apply\Theta \rho}{\Delta} }
  { \synth{\Gamma}{\{\ell : e , r \}}{\{\ell : A, \rho\}}{\Delta} }
\]

\[
  \deduct[(Prj)]
  { \synth{\Gamma}{e}{A}{\Theta} \spc
    \presynth{\Theta}{\apply\Theta A}{\prj}{\ell}{C}{\Delta}
  }
  { \synth{\Gamma}{e\#\ell}{C}{\Delta} }
  \spc
  \deduct[(\(\forall\) Prj)]
  { \presynth{\Gamma, \ev\alpha}{A[\alpha := \ev\alpha]}{\prj}{\ell}{C}{\Delta} }
  { \presynth{\Gamma}{\forall \alpha. A}{\prj}{\ell}{C}{\Delta} }
\]

\[
  \deduct[(RcdPrjR)]
    { \lookup{\Gamma}{R}{l}{C}{\Delta} }
    { \presynth{\Gamma}{R}{\prj}{\ell}{C}{\Delta} }
\]

\noindent
We define record lookup $\lookup{\Gamma}{\rho}{\ell}{A}{\Delta}$ as follows (inputs: $\Gamma$, $\rho$, $l$; outputs: $A$, $\Delta$):
\[
\deduct[(lookupYes)]{}{\lookup{\Gamma}{\{\ell : A, R\}}{\ell}{A}{\Gamma}}
\spc
\deduct[(lookupNo)]
  {\ell \neq \ell' \spc \lookup{\Gamma}{\{R\}}{\ell}{A}{\Delta}}
  {\lookup{\Gamma}{\{\ell' : A', R\}}{\ell}{A}{\Delta}}
\]

\[
\deduct[(Lookup \(\ev\alpha\))]
  { }
  { \lookup
      {\Gamma[\ev\alpha]}
      {\ev\alpha}
      {\ell}
      {\ev\alpha_0}
      {\Gamma[\ev\alpha_0, \ev\alpha_1, \ev\alpha = \{\ell : \ev\alpha_0, \ev\alpha_1\}] }
  }
\]

\[
\deduct[(Lookup RowTail)]
  { }
  { \lookup
      {\Gamma[\ev\alpha]}
      {\{\ev\alpha\}}
      {\ell}
      {\ev\alpha_0}
      {\Gamma[\ev\alpha_0, \ev\alpha_1, \ev\alpha = (\ell : \ev\alpha_0, \ev\alpha_1)] }
  }
\]

\section{Subsumption}
We define the algorithmic subsumption:
\[
\subsumes{\Gamma}{A_0}{A_1}{\Delta}
\]
which represents $A_0$ subsumes $A_1$ with input context $\Gamma$ and output
context $\Delta$. Subsumption is like subtyping, but only applies to
quantifiers. Everything else must be strict equality (for now, this also means
records, so you can't use \(\{\ell_1: \texttt{Bool}, \ell_2: \texttt{Num}\}\) in
place of \(\{\ell_1 : \texttt{Bool}\}\) even though you really \emph{should} be
able to).
\[
  \deduct[(EVar)]{}{\subsumes{\Gamma[\ev\alpha]}{\ev\alpha}{\ev\alpha}{\Gamma[\ev\alpha]}}
  \spc
  \deduct[(Var)]{}{\subsumes{\Gamma[\alpha]}{\alpha}{\alpha}{\Gamma[\alpha]}}
  \spc
  \deduct[(Const)]{}{\subsumes{\Gamma}{\B}{\B}{\Gamma}}
\]

\[
  \deduct[(\(\forall\)L)]
  { \subsumes{\Gamma, \marker{\ev\alpha}, \ev\alpha}{A[\alpha := \ev\alpha]}{B}{\Delta, \marker{\ev\alpha}, \Theta} }
  { \subsumes{\Gamma}{\forall \alpha. A}{B}{\Delta} }
  \spc
  \deduct[(\(\forall\)R)]
  { \subsumes{\Gamma, \alpha}{A}{B}{\Delta, \alpha, \Theta} }
  { \subsumes{\Gamma}{A}{\forall \alpha. B}{\Delta} }
\]

\[
  \deduct[(\(\to\))]
  { \subsumes{\Gamma}{B_1}{A_1}{\Theta} \spc \subsumes{\Theta}{\apply\Theta A_2}{\apply\Theta B_2}{\Delta} }
  { \subsumes{\Gamma}{A_1 \to A_2}{B_1 \to B_2}{\Delta} }
\]

\[
  \deduct[InstantiateL]
  {\ev \alpha \notin FV(A) \spc \instL{\Gamma[\ev \alpha]}{\ev
      \alpha}{A}{\Delta}}
  {\subsumes{\Gamma[\ev \alpha]}{\ev \alpha}{A}{\Delta}}
  \spc
  \deduct[InstantiateR]
  {\ev \alpha \notin FV(A) \spc \instR{\Gamma[\ev \alpha]}{A}{\ev
      \alpha}{\Delta}}
  {\subsumes{\Gamma[\ev \alpha]}{A}{\ev \alpha}{\Delta}}
\]

\[
  \deduct[Record]{\subsumes{\Gamma}{R_0}{R_1}{\Delta}}{\subsumes{\Gamma}{\{R_0\}}{\{R_1\}}{\Delta}}
\]

For this rule, we treat the rows as sets and assume they are reordered so that
the matching labels are at the front of the row. An algorithmic implementation
would want to deal with the recursive and base cases by looking at the set
intersection and difference of the rows.

\[
  \deduct[(Row)]
  {\subsumes{\Gamma}{A}{B}{\Theta}
    \spc
    \subsumes{\Theta}{[\Theta] R_1}{[\Theta] R_2}{\Delta}
  }
  { \subsumes{\Gamma}{\ell : A, R_1}{\ell : B, R_2}{\Delta} }
  \spc
  \deduct[(Row Nil)]{}{ \subsumes{\Gamma}{\cdot}{\cdot}{\Gamma} }
\]

\[
  \deduct[(RowMissingL)]
  {\instL{\Gamma}{\eva}{r}{\Delta}}
  {\subsumes{\Gamma[\eva], \cdot}{\eva}{r}{\Delta}}
  \spc
  \deduct[(RowMissingR)]
  {\instR{\Gamma}{r}{\eva}{\Delta}}
  {\subsumes{\Gamma[\eva], \cdot}{r}{\eva}{\Delta}}
\]

This rule also treats the rows as sets and assumes that there are no equal
labels between the two rows. Assume an analagous rule for a different order of
EVars (the order doesn't matter).
\[
  \deduct[(RowMissingLR)]
  {\subsumes{\Gamma}{\eva}{m_1 : B_1, m_2 : B_2, \dots}{\Theta} \spc
    \subsumes{\Theta}{\ell_1 : \apply\Theta A_1, \ell_2 : \apply\Theta A_2,
      \dots}{\evb}{\delta}}
  {\subsumes{\Gamma[\eva][\evb]}{\ell_1 : A_1, \ell_2 : A_2, \dots, \eva}{m_1 :
      B_1, m_2 : B_2, \dots, \evb}{\Delta}}
\]

\[
  \deduct[(RowTailReach)]
  {}
  {\subsumes{\Gamma[\eva][\evb]}{\eva}{\evb}{\Gamma[\eva][\evb=\eva]}}
\]


\section{Instantiation}

Instantiation is a judgement that solves an EVar, either on the left or right
side of the judgement. It is important to recurisvely assign ``helper'' EVars to
match the shape of whatever the EVar is being instantiated to instead of blindly
assigning it, as there may be EVars inside of the type it is being assigned to.

The EVar is instantiated so that it subsumes or is subsumed by the type,
depending on the type of instantiation (left or right, respectively).

\consider{Need an instantiation for rows that is like InstRcd}

\subsection{Left Instantiation}

\[
  \deduct[InstLSolve]
  {\Gamma \vdash \B}
  {\instL{\Gamma[\ev\alpha]}{\ev \alpha}{\B}{\Gamma[\ev\alpha = \B]}}
  \spc
  \deduct[InstLReach]
  {}
  {\instL{\Gamma[\ev\alpha][\ev\beta]}{\ev \alpha}{\ev
      \beta}{\Gamma[\ev\alpha][\ev\beta = \ev\alpha]}}
\]

\[
  \deduct[InstLArr] {\instR{\Gamma[\eva[2], \eva[1], \eva = \eva[1] \to
      \eva[2]]}{A_1}{\eva[1]}{\Theta} \spc \instL{\Theta}{\eva[2]}{\apply \Theta
      A_2}{\Delta} } {\instL{\Gamma[\eva]}{\eva}{A_1 \to A_2}{\Delta}}
\]

\[
  \deduct[InstLRcd]
  {
    \begin{array}{l}
     \instL{\Gamma_0[(\eva[k+1]), \eva[k], \dots, \eva[1], \eva=\{\ell_1 : \eva[1],
      \ell_2 : \eva[2], \dots, \eva[k], (\eva[k+1])\}]}{\eva[1]}{A_1}{\Gamma_1} \\
    \instL{\Gamma_1}{\eva[2]}{\apply{\Gamma_1} A_2}{\Gamma_2} \spc \cdots \spc
     (\instL{\Gamma_k}{\eva[k+1]}{\apply{\Gamma_k}\evb}{\Delta})
  \end{array}
  }
  {\instL{\Gamma_0[\eva]}{\eva}{\{\ell_1:A_1, \ell_2 : A_2, \dots, \ell_{k-1} : A_{k-1}, (\evb)\}}{\Delta}}
\]
In the above rule, the parentheticals only come into play if there is a row tail
in the record. If there isn't, assume that \(\Delta = \Gamma_k\).

\[
  \deduct[InstLAllR]
  { \instL{\Gamma[\eva], \beta}{\eva}{B}{\Delta, \beta, \Delta'} }
  { \instL{\Gamma[\eva]}{\eva}{\forall \beta. B}{\Delta} }
\]

\subsection{Right Instantiation}

\[
  \deduct[InstRSolve]
  {\Gamma \vdash \B}
  {\instR{\Gamma[\ev\alpha]}{\B}{\ev \alpha}{\Gamma[\ev\alpha = \B]}}
  \spc
  \deduct[InstRReach]
  {}
  {\instR{\Gamma[\ev\alpha][\ev\beta]}{\ev
      \beta}{\eva}{\Gamma[\ev\alpha][\ev\beta = \ev\alpha]}}
\]

\[
  \deduct[InstRArr] {\instL{\Gamma[\eva[2], \eva[1], \eva = \eva[1] \to
      \eva[2]]}{\eva[1]}{A_1}{\Theta} \spc \instR{\Theta}{\apply \Theta
      A_2}{\eva[2]}{\Delta} } {\instR{\Gamma[\eva]}{A_1 \to A_2}{\eva}{\Delta}}
\]

\[
  \deduct[InstRRcd]
  {
    \begin{array}{l}
     \instR{\Gamma_0[(\eva[k+1]), \eva[k], \dots, \eva[1], \eva=\{\ell_1 : \eva[1],
      \ell_2 : \eva[2], \dots, \eva[k], (\eva[k+1])\}]}{A_1}{\eva[1]}{\Gamma_1} \\
    \instR{\Gamma_1}{\apply{\Gamma_2} A_2}{\eva[2]}{\Gamma_3} \spc \cdots \spc
     (\instR{\Gamma_k}{\evb}{\eva[k+1]}{\Delta})
  \end{array}
  }
  {\instR{\Gamma_0[\eva]}{\{\ell_1:A_1, \ell_2 : A_2, \dots, \ell_{k-1} : A_{k-1}, (\evb)\}}{\eva}{\Delta}}
\]
In the above rule, the parentheticals only come into play if there is a row tail
in the record. If there isn't, assume that \(\Delta = \Gamma_k\).

\[
  \deduct[InstRAllL]
  { \subsumes{\Gamma[\eva], \marker{\ev\alpha}, \evb}{A[\beta := \evb]}{\eva}{\Delta, \marker{\evb}, \Delta'} }
  { \subsumes{\Gamma[\eva]}{\forall \beta. B}{\eva}{\Delta} }
\]


\section{Extended Declarative Type System}
An extension of \emph{Complete and Easy}'s declarative type system.

\subsection{Terms}

\[
  \begin{array}{lcl}
    e & \define & x \\
      & | & () \\
      & | & \rcd{} \\
      & | & \lambda x. e  \\
      & | & e e \\
      & | & (e : A) \\
      & | & \rcd{\ell_1 : e_1, \dots, \ell_k : e_k} \\
      & | & (e)
  \end{array}
\]

\subsection{Types}
\[
  \begin{array}{llcl}
    \text{Base types} & \B & \define & 1 \\
                      & & | & \code{Num} \\
                      & & | & \code{Bool} \\
                      & & | & \cdots \\
    \text{Types} & A, B, C & \define & \B \\
                      & & | & \alpha \\
                      & & | & \forall \alpha. A \\
                      & & | & A \to B \\
                      & & | & \rcd{R} \\
    \text{Monotypes} & \tau, \sigma & \define & \B \\
                      & & | & \alpha \\
                      & & | & \tau \to \sigma \\
                      & & | & \rcd{\rho} \\
    \text{Rows} & R & \define & \cdot \\
                      & & | & \rowvar \\
                      & & | & \ell : A, R \\
    \text{Monorows} & \rho & \define & \cdot \\
                      & & | & \ell : \tau, \rho
  \end{array}
\]

\subsection{Contexts}
\[
  \begin{array}{lcl}
    \Psi & \define & \cdot \\
         & | & \Psi, \alpha \\
         & | & \Psi, \rowvar \\
         & | & \Psi, x : A
    
  \end{array}
\]

\subsection{Well-formedness}

The judgement \(\wf \Psi A\) asserts that in the context \(\Psi\), \(A\) is
well-formed. We will make an abuse of notation and have \(\wf \Psi R\) apply to
rows as well, even though they are not really types.

\[
  \deduct[(DeclUVarWF)]
  {\alpha \in \Psi}
  {\wf{\Psi}{\alpha}}
  \spc
  \deduct[(DeclBaseWF)]
  {}
  {\wf{\Psi}{\B}}
\]
\[
  \deduct[(DeclArrowWF)]
  {\wf \Psi A \spc \wf \Psi B}
  {\wf \Psi {A \to B}}
  \spc
  \deduct[(DeclForallWF)]
  {\wf{\Psi, \alpha}{A}}
  {\wf \Psi {\forall \alpha. A}}
  \spc
  \deduct[(DeclRecordWF)]
  {\wf \Psi R}
  {\wf \Psi {\rcd{R}}}
\]

\[
  \deduct[(DeclRowNilWF)]
  {}
  {\wf \Psi \cdot}
  \spc
  \deduct[(DeclRowVarWF)]
  {\rowvar \in \Psi}
  {\wf \Psi \rowvar}
  \spc
  \deduct[(DeclRowWF)]
  {\wf \Psi A \spc \wf \Psi R}
  {\wf \Psi {\ell : A, R}}
\]

% Error in Complete and Easy
% \[
%   \begin{prooftree}
%     \hypo{\beta \in \cdot, \beta}
%     \infer1{\wf{\cdot, \beta}{\beta \le \beta}}
%     \infer1{\wf{\cdot}{\beta \le \forall \beta. \beta}}
%   \end{prooftree}
% \]
% By Lemma 4, this implies \(\wf{\cdot}{\beta}\) which is not true.

\subsection{Subtyping}
We'll also make an abuse of notation and have the judgement \(\subtypes{\Psi}{R_1}{R_2}\) be valid for subtyping.

TODO: add the rules from complete and easy.
\[
  \deduct[(\(\subtype\) RowNil)]
  {}
  {\subtypes{\Psi}{\cdot}{\cdot}}
  \spc
  \deduct[(\(\subtype\) RowVar)]
  {}
  {\subtypes{\Psi}{\rowvar}{\rowvar}}
\]

We treat the rows sort of as sets and reorder the labels appropriately to match
them.

\[
  \deduct[(\(\subtype\) Row)]
  {\subtypes{\Psi}{A}{B} \spc \subtypes{\Psi}{R_1}{R_2}}
  {\subtypes \Psi {\ell : A, R_1}{\ell : B, R_2}}
  \spc
  \deduct[(\(\subtype\) \text{Rcd})]
  {\subtypes \Psi R_1 R_2}
  {\subtypes {\Psi} {\rcd{R_1}} {\rcd{R_2}}}
  \deduct[(\(\subtype\forall\)L Rcd)]
  {\wf \Psi \rho \spc \subtypes \Psi {[\alpha \mapsto \rho]A} B }
  {\wf \Psi }
\]

\end{document}