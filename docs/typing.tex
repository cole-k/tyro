\documentclass{article}

% \prooftree and related commands
\usepackage{ebproof}
% most symbols
\usepackage{amsmath}
\usepackage{amssymb}
% graphic includes + rotation
\usepackage{graphicx}
% better font
\usepackage{lmodern}
\usepackage{upgreek}
% theorems
\usepackage{amsthm}
% \RHD
\usepackage[nointegrals]{wasysym}
\usepackage{xcolor}

% change font
\renewcommand{\familydefault}{\sfdefault}

% remove indent
\setlength\parindent{0pt}

% theorems
\newtheorem{thm}{Theorem}
\newtheorem{prop}[thm]{Proposition}
\newtheorem{lem}[thm]{Lemma}

% https://tex.stackexchange.com/a/391446/132298
\newtheorem{manuallemmainner}{Lemma}
\newenvironment{manuallemma}[1]{%
  \renewcommand\themanuallemmainner{#1}%
  \manuallemmainner
}{\endmanuallemmainner}

\newcommand{\ignore}[1]{}
\newcommand{\todo}[1]{\colorbox{red}{#1}}
\newcommand{\consider}[1]{\colorbox{red}{!!!} #1 \colorbox{red}{!!!}}
\newcommand{\code}[1]{\texttt{#1}}

\renewcommand{\implies}{\Rightarrow}
\renewcommand{\impliedby}{\Leftarrow}
\newcommand{\declCtx}{\Uppsi}
\newcommand{\rcd}[1]{\{#1\}}
\newcommand{\B}{\mathcal{B}}
\newcommand{\C}{\mathcal{C}}
\newcommand{\D}{\mathcal{D}}
\newcommand{\rowall}{\rotatebox[origin=c]{180}{\(\mathsf{R}\)}}
\newcommand{\rowvar}{\alpha_\rho}
\newcommand{\bottom}{\perp}
\newcommand{\define}{::=}
\newcommand{\marker}[1]{\RHD_{#1}}
\newcommand{\subsume}{<:}
\newcommand{\synthesizes}{\Rightarrow \!\!\! \Rightarrow}
\newcommand{\app}{\bullet}
\newcommand{\prjSymbol}{.}
\newcommand{\prj}{\,\prjSymbol\,}
\newcommand{\instLSymbol}{\;\substack{<\\:=}\;}
\newcommand{\instRSymbol}{\;\substack{<\\=:}\;}
\newcommand{\ev}{\hat}
\newcommand{\spc}{\qquad}
\newcommand{\apply}[1]{\left[#1\right]}
\newcommand{\eva}[1][]{\ev \alpha_{#1}}
\newcommand{\evb}{\ev \beta}

\newcommand{\rnil}{\{\}}
\newcommand{\rcons}[2]{\{#1 \,|\, #2\}}

\newcommand{\wf}[2]{#1 \vdash #2}

\newcommand{\subtype}{\le}
\newcommand{\subtypes}[3]{#1 \vdash #2 \le #3}

\newcommand{\declSynth}[3]{#1 \vdash #2 \Rightarrow #3}
\newcommand{\declCheck}[3]{#1 \vdash #2 \Leftarrow #3}
\newcommand{\declApSynth}[4]{#1 \vdash #2 \app #3 \synthesizes #4}
\newcommand{\declLookup}[4]{#1 \vdash #2 \# #3 \longrightarrow #4}

\newcommand{\synth}[4]{#1 \vdash #2 \Rightarrow #3 \dashv #4}
\renewcommand{\check}[4]{#1 \vdash #2 \Leftarrow #3 \dashv #4}

\newcommand{\presynth}[6]{#1 \vdash #2 #3 #4 \synthesizes #5 \dashv #6}

\newcommand{\subsumes}[4]{#1 \vdash #2 \subsume #3 \dashv #4}

\newcommand{\instL}[4]{#1 \vdash #2 \instLSymbol #3 \dashv #4}
\newcommand{\instR}[4]{#1 \vdash #2 \instRSymbol #3 \dashv #4}

\newcommand{\lookup}[5]{#1 \vdash #2 \# #3 \longrightarrow #4 \dashv #5}

\newcommand{\extends}{\longrightarrow}

\newcommand{\deduct}[3][]
{
  \begin{prooftree}
    \hypo{#2}
    \infer1[\text{#1}]{#3}
  \end{prooftree}
}

\begin{document}

\section{Introduction}
\consider{These rules are WIP - the ones that come directly from
  \textit{Complete and Easy} should be right, but the rest have not been proven}

We present a type inference and checking system based off of the one in
\textit{Complete and Easy Bidirectional Typechecking for Higher-Rank
  Polymorphism} by Dunfield and Krishnaswami. This system is extended to
encompass record types.

Future work may include extending to full(er) subtyping, as well as variant
types.

\section{Conventions}

We sometimes use $\beta$ and $\gamma$ for type variables in addition
to $\alpha$. We adopt the same
conventions Dunfield and Krishnaswami use for their typing rules.

In record types, we elide the final empty row type
($\cdot$); e.g. we write $\{\ell_0 : A_0, \ell_1 : A_1\}$ instead of $\{\ell_0 : A_0, \ell_1 : A_1, \cdot\}$.

% We treat rows sort of like sets, even though we represent them for simplicity's
% sake as cons lists in the Grammar. Whenever we have syntax that looks like
% \[\subsumes{\Gamma}{\ell_1 : A_1, \ell_2 : A_2, \dots}{\ell_1 : B_1, \ell_2 :
%     B_2, \dots, m_1 : B_1', m_2 : B_2', \dots}{\Delta}\]
% We are implicitly reordering the sets so that the labels that match are at the
% front and the mismatched labels are at the back. This is important for the row
% subsumption rules, since it means that we can move the mismatched label
% resolution to a base case. An implementation of these rules would presumably use
% something easier, like set difference, to figure out which labels to compare.


\section{Extended Declarative Type System}
An extension of \emph{Complete and Easy}'s declarative type system.

\subsection{Terms}

\[
  \begin{array}{llcll}
    \text{Terms} & e & \define & x & \text{Variable} \\
                 & & | & () & \text{1} \\
                 & & | & \rcd{} & \text{Empty record}\\
                 & & | & \lambda x \to e  & \text{Lambda} \\
                 & & | & e\ e & \text{Application} \\
                 & & | & (e : A) & \text{Type annotation}\\
                 & & | & \rcd{\ell_1 : e_1, \dots, \ell_k : e_k} & \text{Record}
    \\
                 & & | & e \prjSymbol \ell & \text{(Record) Projection} \\
                 & & | & (e) & \text{Parentheses} \\
   \text{Labels} & \ell & \define & \{abcdefghijklmnopqrstuvwxyz\}^+
  \end{array}
\]

\subsection{Types}
\[
  \begin{array}{llcl}
    \text{Base types} & \B & \define & 1 \\
                      & & | & \code{Num} \\
                      & & | & \code{Bool} \\
                      & & | & \cdots \\
    \text{Types} & A, B, C & \define & \B \\
                      & & | & \alpha \\
                      & & | & \forall \alpha. A \\
                      & & | & \rowall \rowvar. A \\
                      & & | & A \to B \\
                      & & | & \rcd{R} \\
    \text{Monotypes} & \tau, \sigma & \define & \B \\
                      & & | & \alpha \\
                      & & | & \tau \to \sigma \\
                      & & | & \rcd{\rho} \\
    \text{Rows} & R & \define & \cdot \\
                      & & | & \rowvar \\
                      & & | & \ell : A, R \\
    \text{Monorows} & \rho & \define & \cdot \\
                      & & | & \ell : \tau, \rho
  \end{array}
\]

\subsubsection{A note on quantifiers}
We distinguish \(\forall\) from \(\rowall\), where the former is a quantifier
over type variables and the latter is a quantifier over row variables.

\subsection{Contexts}
\[
  \begin{array}{lcl}
    \declCtx & \define & \cdot \\
         & | & \declCtx, \alpha \\
         & | & \declCtx, \rowvar \\
         & | & \declCtx, x : A
    
  \end{array}
\]

\subsection{Well-formedness}

The judgement \(\wf \declCtx A\) asserts that in the context \(\declCtx\), \(A\) is
well-formed. We will make an abuse of notation and have \(\wf \declCtx R\) apply to
rows as well, even though they are not really types.

\[
  \deduct[(DeclUVarWF)]
  {\alpha \in \declCtx}
  {\wf{\declCtx}{\alpha}}
  \spc
  \deduct[(DeclBaseWF)]
  {}
  {\wf{\declCtx}{\B}}
  \spc
  \deduct[(DeclArrowWF)]
  {\wf \declCtx A \spc \wf \declCtx B}
  {\wf \declCtx {A \to B}}
\]
\[
  \deduct[(DeclForallWF)]
  {\wf{\declCtx, \alpha}{A}}
  {\wf \declCtx {\forall \alpha. A}}
  \spc
  \deduct[(DeclForallRowWF)]
  {\wf{\declCtx, \rowvar}{A}}
  {\wf \declCtx {\rowall \rowvar. A}}
  \spc
  \deduct[(DeclRecordWF)]
  {\wf \declCtx R}
  {\wf \declCtx {\rcd{R}}}
\]

\[
  \deduct[(DeclRowNilWF)]
  {}
  {\wf \declCtx \cdot}
  \spc
  \deduct[(DeclRowVarWF)]
  {\rowvar \in \declCtx}
  {\wf \declCtx \rowvar}
  \spc
  \deduct[(DeclRowWF)]
  {\wf \declCtx A \spc \wf \declCtx R}
  {\wf \declCtx {\ell : A, R}}
\]

\subsection{Subtyping}
We'll also make an abuse of notation and have the judgement \(\subtypes{\declCtx}{R_1}{R_2}\) be valid for subtyping.

TODO: add the rules from complete and easy.
\[
  \deduct[(\(\subtype\) RowNil)]
  {}
  {\subtypes{\declCtx}{\cdot}{\cdot}}
  \spc
  \deduct[(\(\subtype\) RowVar)]
  {\rowvar \in \declCtx}
  {\subtypes{\declCtx}{\rowvar}{\rowvar}}
\]

We treat the rows sort of as sets and reorder the labels appropriately to match
them.

\[
  \deduct[(\(\subtype\) Row)]
  {\subtypes{\declCtx}{A}{B} \spc \subtypes{\declCtx}{R_1}{R_2}}
  {\subtypes \declCtx {\ell : A, R_1}{\ell : B, R_2}}
  \spc
  \deduct[(\(\subtype\) \text{Rcd})]
  {\subtypes \declCtx R_1 R_2}
  {\subtypes {\declCtx} {\rcd{R_1}} {\rcd{R_2}}}
\]
\[
  \deduct[(\(\subtype\rowall\)L Rcd)]
  {\wf \declCtx \rho \spc \subtypes \declCtx {[\rowvar \mapsto \rho]\rcd{R_1}} {\rcd{R_2}} }
  {\wf \declCtx \rowall \rowvar. \rcd{R_1} \le \rcd{R_2}}
  \spc
  \deduct[(\(\subtype\rowall\)R Rcd)]
  {\wf \declCtx {\rcd{R_1}} \spc \subtypes {\declCtx,\beta} {\rcd{R_1}} {\rcd{R_2}} }
  {\subtypes \declCtx {\rcd{R_1}} {\rowall \rowvar. \rcd{R_2}} }
\]

\subsection{Declarative Typing}

\subsubsection{Judgements}

\(\declCheck \declCtx e A\), Under context \(\declCtx\), \(e\) checks against
type \(A\).

\(\declSynth \declCtx e A\), Under context \(\declCtx\), \(e\) synthesizes type
\(A\).

\(\declApSynth \declCtx A e C\), Under context \(\declCtx\) type \(A\) applied
to \(e\) synthesizes type \(C\).

\(\declLookup \declCtx A \ell C\), under context \(\declCtx\) type \(A\)
projects type \(C\) at \(\ell\).

\subsubsection{Rules}

\consider{All of the declarative rules from \emph{Complete and Easy} still
  apply, though they have yet to be added.}

\[
  \deduct[(Decl RcdI)]
  {\declCheck \declCtx e A \spc \declCheck \declCtx r \rcd{R}}
  {\declCheck \declCtx {\rcd{\ell : e, r}} {\rcd{\ell : A, R}}}
  \spc
  \deduct[(Decl RcdI \(\Rightarrow\))]
  {\declSynth \declCtx e A \spc \declSynth \declCtx r \rcd{R}}
  {\declSynth \declCtx {\rcd{\ell : e, r}} {\rcd{\ell : A, R}}}
\]

\[
  \deduct[(Decl \(\rowall\)I)]
  {\declCheck {\declCtx, \rowvar} e A}
  {\declCheck \declCtx e {\rowall \rowvar. A}}
  \spc
  \deduct[(DeclPrjI)]
  {\declSynth \declCtx e A \spc \declLookup \declCtx A \ell C}
  {\declSynth \declCtx {e.\ell} C}
\]

\[
  \deduct[(Decl Lookup)]
  {}
  {\declLookup \declCtx {\rcd{\ell : C, R}} \ell C}
  \spc
  \deduct[(Decl \(\rowall\) Lookup)]
  {\wf \declCtx \rho \spc \declLookup \declCtx {[\rowvar \mapsto \rho]A} \ell C}
  {\declLookup \declCtx {\rowall \rowvar. A} \ell C}
\]

\[
  \deduct[(Decl \(\forall\) Lookup)]
  {\wf \declCtx \tau \spc \declLookup \declCtx {[\alpha \mapsto \tau]A} \ell C}
  {\declLookup \declCtx {\forall \alpha. A} \ell C}
\]


\section{Algorithmic Typing}

\subsection{Notes}

The algorithmic typing, unlike the declarative typing, presently doesn't
distinguish between rowvars and evars. The two are very similar in nature, and
so we conflate them in the typing rules. It's just that rowvars can only be
resolved to rows. This subtle difference is partially reflected in the
implementaiton of the language: rowvars and evars are separate; however,
quantifiers are not.

For convenience's sake, these rules will not presently distinguish between
rowvars and evars, although this may change.

\subsection{Terms}

Terms are the same as in the declarative system.

\subsection{Types}

The main difference in types for the Algorithmic System is the addition of evars
(\(\eva\)). There are some notational differences too, as a result.

\[\begin{array}{rcl}
\B & \define & \texttt{Num} \mid \texttt{Bool} \mid \cdots
\\
% c & \define & \bottom \mid \tau \mid \top
% \\
A & \define & \B \mid \ev\alpha \mid \alpha \mid \forall \alpha. A \mid A_0 \to A_1 \mid \{R\}
\\
\tau & \define & \B \mid \ev\alpha \mid \alpha \mid \tau_0 \to \tau_1 \mid \{\rho\}
\\
R & \define & \cdot \mid \ev\alpha \mid \ell : A, R
\\
r & \define & \cdot \mid \ell : A, r
\\
\rho & \define & \cdot \mid \ev\alpha \mid \ell : \tau, \rho
\end{array}
\]

\subsection{Typing Rules}
We define algorithmic typing with the following judgments:
\[
\check{\Gamma}{e}{A}{\Delta}
\spc
\synth{\Gamma}{e}{A}{\Delta}
\]
which respectively represent type checking (inputs: $\Gamma$, $e$, $A$; output: $\Delta$) and type synthesis (inputs: $\Gamma$, $e$; outputs: $A$, $\Delta$).

We also define a binary algorithmic judgement:

\[
\presynth{\Gamma}{X}{\square}{Y}{Z}{\Delta}
\]

which represents a binary judgement \(\square\) under the context \(\Gamma\) on
values \(X\) and \(Y\) that synthesizes \(Z\) with output context \(\Delta\).
For example, the syntax that Dunfield and Krishnaswami use for function
application synthesis judgements would be

\[
\presynth \Gamma A \app e C \Delta
\]

which means that under context \(\Gamma\), \(A\) applied to the term \(e\)
synthesizes output type \(C\) and context \(\Delta\).

\[
  \deduct[(Var)]
  {
    (x : A) \in \Gamma
  }
  { \synth{\Gamma}{x}{A}{\Gamma} }
  \spc
  \deduct[(Sub)]
  {
    \synth{\Gamma}{e}{A}{\Theta} \spc
    \subsumes{\Theta}{\apply\Theta A}{\apply\Theta B}{\Delta}
  }
  { \check{\Gamma}{e}{B}{\Delta} }
\]

\[
  \deduct[(Annotation)]
  { \Gamma \vdash A \spc \check{\Gamma}{e}{A}{\Delta} }
  { \synth{\Gamma}{(e : A)}{A}{\Delta} }
  \spc
  \deduct[(\(\forall\) I)]
  { \check{\Gamma, \alpha}{e}{A}{\Delta, \alpha, \Theta} }
  { \check{\Gamma}{e}{\forall \alpha. A}{\Delta} }
\]

\[
  \deduct[(\(\to\)I)]
  { \check{\Gamma, x : A}{e}{B}{\Delta, x : A, \Theta} }
  { \check{\Gamma}{\lambda x. e}{A \to B}{\Delta} }
  \spc
  \deduct[(\(\to\)I\(\implies\))]
  { \check{\Gamma, \ev\alpha, \ev\beta, x : \ev\alpha}{e}{\ev\beta}{\Delta, x : \ev\alpha, \Theta} }
  { \synth{\Gamma}{\lambda x. e}{\ev\alpha \to \ev\beta}{\Delta} }
\]

\[
  \deduct[(\(\to\)E)]
  { \synth{\Gamma}{e_1}{A}{\Theta} \spc \presynth{\Theta}{\apply\Theta A}{\app}{e_2}{C}{\Delta} }
  { \synth{\Gamma}{e_1 \ e_2}{C}{\Delta}  }
  \spc
  \deduct[(\(\forall\) App)]
  { \presynth{\Gamma, \ev\alpha}{A[\alpha := \ev\alpha]}{\app}{e}{C}{\Delta} }
  { \presynth{\Gamma}{\forall \alpha. A}{\app}{e}{C}{\Delta} }
\]

\[
  \deduct[(\(\to\)App)]
    { \check{\Gamma}{e}{A}{\Delta} }
    { \presynth{\Gamma}{A \to C}{\app}{e}{C}{\Delta} }
\]

\[
  \deduct[(\(\ev\alpha\)App)]
    {
      \check{\Gamma[\ev{\alpha_2}, \ev{\alpha_1}, \ev \alpha = \ev{\alpha_1} \to
        \ev{\alpha_2}]}{e}{\ev{\alpha_1}}{\Delta}
    }
    { \presynth{\Gamma[\ev\alpha]}{\ev\alpha}{\app}{e}{\ev\alpha_2}{\Delta} }
\]


\[
\deduct[(\{\}I)]{}{\check{\Gamma}{\{\}}{\{\}}{\Gamma}}
\spc
\deduct[(\{\}I\(\implies\))]{}{\synth{\Gamma}{\{\}}{\{\}}{\Gamma}}
\]

\[
  \deduct[(RcdI)]
  { \check{\Gamma}{e}{A}{\Theta} \spc \check{\Theta}{\apply\Theta r}{\apply\Theta\rho}{\Delta} }
  { \check{\Gamma}{\{\ell : e , r \}}{\{\ell : A, \rho\}}{\Delta} }
  \spc
  \deduct[(RcdI\(\implies\))]
  { \synth{\Gamma}{e}{A}{\Theta} \spc \synth{\Theta}{\apply\Theta r}{\apply\Theta \rho}{\Delta} }
  { \synth{\Gamma}{\{\ell : e , r \}}{\{\ell : A, \rho\}}{\Delta} }
\]

\[
  \deduct[(Prj)]
  { \synth{\Gamma}{e}{A}{\Theta} \spc
    \presynth{\Theta}{\apply\Theta A}{\prj}{\ell}{C}{\Delta}
  }
  { \synth{\Gamma}{e\#\ell}{C}{\Delta} }
  \spc
  \deduct[(\(\forall\) Prj)]
  { \presynth{\Gamma, \ev\alpha}{A[\alpha := \ev\alpha]}{\prj}{\ell}{C}{\Delta} }
  { \presynth{\Gamma}{\forall \alpha. A}{\prj}{\ell}{C}{\Delta} }
\]

\[
  \deduct[(RcdPrjR)]
    { \lookup{\Gamma}{R}{l}{C}{\Delta} }
    { \presynth{\Gamma}{R}{\prj}{\ell}{C}{\Delta} }
\]

\noindent
We define record lookup $\lookup{\Gamma}{\rho}{\ell}{A}{\Delta}$ as follows (inputs: $\Gamma$, $\rho$, $l$; outputs: $A$, $\Delta$):
\[
\deduct[(lookupYes)]{}{\lookup{\Gamma}{\{\ell : A, R\}}{\ell}{A}{\Gamma}}
\spc
\deduct[(lookupNo)]
  {\ell \neq \ell' \spc \lookup{\Gamma}{\{R\}}{\ell}{A}{\Delta}}
  {\lookup{\Gamma}{\{\ell' : A', R\}}{\ell}{A}{\Delta}}
\]

\[
\deduct[(Lookup \(\ev\alpha\))]
  { }
  { \lookup
      {\Gamma[\ev\alpha]}
      {\ev\alpha}
      {\ell}
      {\ev\alpha_0}
      {\Gamma[\ev\alpha_0, \ev\alpha_1, \ev\alpha = \{\ell : \ev\alpha_0, \ev\alpha_1\}] }
  }
\]

\[
\deduct[(Lookup RowTail)]
  { }
  { \lookup
      {\Gamma[\ev\alpha]}
      {\{\ev\alpha\}}
      {\ell}
      {\ev\alpha_0}
      {\Gamma[\ev\alpha_0, \ev\alpha_1, \ev\alpha = (\ell : \ev\alpha_0, \ev\alpha_1)] }
  }
\]

\subsection{Subsumption}
We define the algorithmic subsumption:
\[
\subsumes{\Gamma}{A_0}{A_1}{\Delta}
\]
which represents $A_0$ subsumes $A_1$ with input context $\Gamma$ and output
context $\Delta$. Subsumption is like subtyping, but only applies to
quantifiers. Everything else must be strict equality (for now, this also means
records, so you can't use \(\{\ell_1: \texttt{Bool}, \ell_2: \texttt{Num}\}\) in
place of \(\{\ell_1 : \texttt{Bool}\}\) even though you really \emph{should} be
able to).
\[
  \deduct[(EVar)]{}{\subsumes{\Gamma[\ev\alpha]}{\ev\alpha}{\ev\alpha}{\Gamma[\ev\alpha]}}
  \spc
  \deduct[(Var)]{}{\subsumes{\Gamma[\alpha]}{\alpha}{\alpha}{\Gamma[\alpha]}}
  \spc
  \deduct[(Const)]{}{\subsumes{\Gamma}{\B}{\B}{\Gamma}}
\]

\[
  \deduct[(\(\forall\)L)]
  { \subsumes{\Gamma, \marker{\ev\alpha}, \ev\alpha}{A[\alpha := \ev\alpha]}{B}{\Delta, \marker{\ev\alpha}, \Theta} }
  { \subsumes{\Gamma}{\forall \alpha. A}{B}{\Delta} }
  \spc
  \deduct[(\(\forall\)R)]
  { \subsumes{\Gamma, \alpha}{A}{B}{\Delta, \alpha, \Theta} }
  { \subsumes{\Gamma}{A}{\forall \alpha. B}{\Delta} }
\]

\[
  \deduct[(\(\to\))]
  { \subsumes{\Gamma}{B_1}{A_1}{\Theta} \spc \subsumes{\Theta}{\apply\Theta A_2}{\apply\Theta B_2}{\Delta} }
  { \subsumes{\Gamma}{A_1 \to A_2}{B_1 \to B_2}{\Delta} }
\]

\[
  \deduct[InstantiateL]
  {\ev \alpha \notin FV(A) \spc \instL{\Gamma[\ev \alpha]}{\ev
      \alpha}{A}{\Delta}}
  {\subsumes{\Gamma[\ev \alpha]}{\ev \alpha}{A}{\Delta}}
  \spc
  \deduct[InstantiateR]
  {\ev \alpha \notin FV(A) \spc \instR{\Gamma[\ev \alpha]}{A}{\ev
      \alpha}{\Delta}}
  {\subsumes{\Gamma[\ev \alpha]}{A}{\ev \alpha}{\Delta}}
\]

\[
  \deduct[Record]{\subsumes{\Gamma}{R_0}{R_1}{\Delta}}{\subsumes{\Gamma}{\{R_0\}}{\{R_1\}}{\Delta}}
\]

For this rule, we treat the rows as sets and assume they are reordered so that
the matching labels are at the front of the row. An algorithmic implementation
would want to deal with the recursive and base cases by looking at the set
intersection and difference of the rows.

\[
  \deduct[(Row)]
  {\subsumes{\Gamma}{A}{B}{\Theta}
    \spc
    \subsumes{\Theta}{[\Theta] R_1}{[\Theta] R_2}{\Delta}
  }
  { \subsumes{\Gamma}{\ell : A, R_1}{\ell : B, R_2}{\Delta} }
  \spc
  \deduct[(Row Nil)]{}{ \subsumes{\Gamma}{\cdot}{\cdot}{\Gamma} }
\]

\[
  \deduct[(RowMissingL)]
  {\instL{\Gamma[\eva]}{\eva}{r}{\Delta}}
  {\subsumes{\Gamma[\eva], \cdot}{\eva}{r}{\Delta}}
  \spc
  \deduct[(RowMissingR)]
  {\instR{\Gamma[\eva]}{r}{\eva}{\Delta}}
  {\subsumes{\Gamma[\eva], \cdot}{r}{\eva}{\Delta}}
\]

This rule also treats the rows as sets and assumes that there are no equal
labels between the two rows. Assume an analagous rule for a different order of
EVars (the order doesn't matter).
\[
  \deduct[(RowMissingLR)]
  {\subsumes{\Gamma}{\eva}{m_1 : B_1, m_2 : B_2, \dots}{\Theta} \spc
    \subsumes{\Theta}{\ell_1 : \apply\Theta A_1, \ell_2 : \apply\Theta A_2,
      \dots}{\evb}{\Delta}}
  {\subsumes{\Gamma[\eva][\evb]}{\ell_1 : A_1, \ell_2 : A_2, \dots, \eva}{m_1 :
      B_1, m_2 : B_2, \dots, \evb}{\Delta}}
\]

\[
  \deduct[(RowTailReach)]
  {}
  {\subsumes{\Gamma[\eva][\evb]}{\eva}{\evb}{\Gamma[\eva][\evb=\eva]}}
\]


\subsection{Instantiation}

Instantiation is a judgement that solves an EVar, either on the left or right
side of the judgement. It is important to recurisvely assign ``helper'' EVars to
match the shape of whatever the EVar is being instantiated to instead of blindly
assigning it, as there may be EVars inside of the type it is being assigned to.

The EVar is instantiated so that it subsumes or is subsumed by the type,
depending on the type of instantiation (left or right, respectively).

\consider{Need an instantiation for rows that is like InstRcd}

\subsubsection{Left Instantiation}

\[
  \deduct[InstLSolve]
  {\Gamma \vdash \B}
  {\instL{\Gamma[\ev\alpha]}{\ev \alpha}{\B}{\Gamma[\ev\alpha = \B]}}
  \spc
  \deduct[InstLReach]
  {}
  {\instL{\Gamma[\ev\alpha][\ev\beta]}{\ev \alpha}{\ev
      \beta}{\Gamma[\ev\alpha][\ev\beta = \ev\alpha]}}
\]

\[
  \deduct[InstLArr] {\instR{\Gamma[\eva[2], \eva[1], \eva = \eva[1] \to
      \eva[2]]}{A_1}{\eva[1]}{\Theta} \spc \instL{\Theta}{\eva[2]}{\apply \Theta
      A_2}{\Delta} } {\instL{\Gamma[\eva]}{\eva}{A_1 \to A_2}{\Delta}}
\]

\[
  \deduct[InstLRcd]
  {
    \begin{array}{l}
     \instL{\Gamma_0[(\eva[k+1]), \eva[k], \dots, \eva[1], \eva=\{\ell_1 : \eva[1],
      \ell_2 : \eva[2], \dots, \ell_k : \eva[k], (\eva[k+1])\}]}{\eva[1]}{A_1}{\Gamma_1} \\
    \instL{\Gamma_1}{\eva[2]}{\apply{\Gamma_1} A_2}{\Gamma_2} \spc \cdots \spc
     (\instL{\Gamma_k}{\eva[k+1]}{\apply{\Gamma_k}\evb}{\Delta})
  \end{array}
  }
  {\instL{\Gamma_0[\eva]}{\eva}{\{\ell_1:A_1, \ell_2 : A_2, \dots, \ell_{k-1} : A_{k-1}, (\evb)\}}{\Delta}}
\]
In the above rule, the parentheticals only come into play if there is a row tail
in the record. If there isn't, assume that \(\Delta = \Gamma_k\).

\[
  \deduct[InstLAllR]
  { \instL{\Gamma[\eva], \beta}{\eva}{B}{\Delta, \beta, \Delta'} }
  { \instL{\Gamma[\eva]}{\eva}{\forall \beta. B}{\Delta} }
\]

\subsubsection{Right Instantiation}

\[
  \deduct[InstRSolve]
  {\Gamma \vdash \B}
  {\instR{\Gamma[\ev\alpha]}{\B}{\ev \alpha}{\Gamma[\ev\alpha = \B]}}
  \spc
  \deduct[InstRReach]
  {}
  {\instR{\Gamma[\ev\alpha][\ev\beta]}{\ev
      \beta}{\eva}{\Gamma[\ev\alpha][\ev\beta = \ev\alpha]}}
\]

\[
  \deduct[InstRArr] {\instL{\Gamma[\eva[2], \eva[1], \eva = \eva[1] \to
      \eva[2]]}{\eva[1]}{A_1}{\Theta} \spc \instR{\Theta}{\apply \Theta
      A_2}{\eva[2]}{\Delta} } {\instR{\Gamma[\eva]}{A_1 \to A_2}{\eva}{\Delta}}
\]

\[
  \deduct[InstRRcd]
  {
    \begin{array}{l}
     \instR{\Gamma_0[(\eva[k+1]), \eva[k], \dots, \eva[1], \eva=\{\ell_1 : \eva[1],
      \ell_2 : \eva[2], \dots, \ell_k : \eva[k], (\eva[k+1])\}]}{A_1}{\eva[1]}{\Gamma_1} \\
    \instR{\Gamma_1}{\apply{\Gamma_2} A_2}{\eva[2]}{\Gamma_3} \spc \cdots \spc
     (\instR{\Gamma_k}{\evb}{\eva[k+1]}{\Delta})
  \end{array}
  }
  {\instR{\Gamma_0[\eva]}{\{\ell_1:A_1, \ell_2 : A_2, \dots, \ell_{k-1} : A_{k-1}, (\evb)\}}{\eva}{\Delta}}
\]
In the above rule, the parentheticals only come into play if there is a row tail
in the record. If there isn't, assume that \(\Delta = \Gamma_k\).

\[
  \deduct[InstRAllL]
  { \subsumes{\Gamma[\eva], \marker{\ev\alpha}, \evb}{A[\beta := \evb]}{\eva}{\Delta, \marker{\evb}, \Delta'} }
  { \subsumes{\Gamma[\eva]}{\forall \beta. B}{\eva}{\Delta} }
\]


\section{Proof Extensions}

We only provide the cases pertaining to the extensions, as the rest of the proof
remains the same as in \textit{Complete and Easy}.

Note: we'll be sort of conflating row variables (\(\rowvar\)) with type
variables (\(\alpha\)) because they are symmetric and also disjoint. We do our
best to visually distinguish the two, but when we prove something involving type
variables, we mean for both type and row variables unless otherwise noted. The
same is true of \(\tau\) and \(\rho\).

\subsection{Properties of Well-Formedness}
\begin{prop}[Weakening]
  If \(\wf \declCtx A\), then \(\wf {\declCtx, \declCtx'} A\) by a derivation of
  the same size.
\end{prop}

This still follows by straightforward induction on the cases we have added.

\begin{prop}[Substitution]
  If \(\wf \declCtx A\) and \(\wf {\declCtx, \alpha, \declCtx'} B\), then \(\wf
  {\declCtx, [\alpha \mapsto A] \declCtx'} {[\alpha \mapsto A] B}\).
\end{prop}

\begin{proof}
The proof is by induction on \(B\).
\begin{itemize}
\item Case \(B = R\).

  If \(\rowvar\) isn't the row tail of \(R\) (i.e. is not at the ``surface
  level'' of the row), then it follows from the inductive hypothesis that \(\wf
  {\declCtx, \declCtx'} R\).

  If \(\rowvar\) does occur in \(R\), then the substitution extends \(R\) with
  the row \(A\). Suppose \(A = \ell_1: A_1, \dots, \ell_k: A_k\). Because \(\wf
  \declCtx A\), we know that by DeclRowWF, \(\wf \declCtx A_i\) for \(1 \le i
  \le k\). By proposition 1, \(\wf {\declCtx, \declCtx'} A_i\) for \(1 \le i \le
  k\), and thus by DeclRowWF \(\wf {\declCtx, \declCtx'} [\rowvar \mapsto A]B\).

\item Case \(B = \rowall \rowvar. A\).

  This follows immediately by the inductive hypothesis after applying
  DeclForallRowWF.

\item Case \(B = \rcd{R}\).

  This follows from Case \(B = R\).
\end{itemize}
\end{proof}

\begin{manuallemma}{2.5}[Row Formedness]
  If \(\wf \declCtx {\ell_1 : A_1, \dots, \ell_k : A_k}\), then \(\wf \declCtx
  {\ell_1 : A_1, \dots, \ell_i : A_i}\) and \(\wf \declCtx A_i\) for all \(1 \le
  i \le k\).
\end{manuallemma}

This follows from DeclRowWF by induction.

\subsection{Reflexivity}
\begin{lem}[Reflexivity of Declarative Subtyping]
  Subtyping is reflexive: if \(\wf \declCtx A \) then \(\subtypes \declCtx A
  A\).
\end{lem}

\begin{proof}
The proof is by induction on \(A\).

\begin{itemize}
\item Case \(A = \rowall \rowvar. A_0\). This follows by the same reasoning for
  \(\forall\).
\item Case \(A = \ell_1 : A_1, \dots, \ell_k : A_k\). By the inductive
  hypothesis, \(\ell_1 : A_1, \dots, \ell_{k-1} : A_{k-1}\) subtypes itself.
  Also by the inductive hypothesis, \(\ell_k : A_k \le \ell_k : A_k\). These
  combined with \(\le\) Row complete the proof.
\item Case \(A = \rcd{R}\). This follows from the row case.
\end{itemize}
  
\end{proof}

\subsection{Subtyping Implies Well-Formedness}

\begin{lem}[Well-Formedness]
  If \(\subtypes \declCtx A B\) then \(\wf \declCtx A\) and \(\wf \declCtx B\).
\end{lem}

\begin{proof}
  The proof is by induction on the derivation.
  \begin{itemize}
  \item Case \(\subtypes \declCtx {\ell : A, R_1} {\ell : B, R_2}\).

    By the hypothesis, \(\subtypes \declCtx A B\) and \(\subtypes \declCtx R_1
    R_2\). From the inductive hypothesis, we get \(\wf \declCtx {A; B; R_1;
      R_2}\). We may then apply DeclRowWF to conclude that \(\wf \declCtx {\ell
      : A, R_1}\) and \(\wf \declCtx {\ell : B, R_2}\), as desired.
  \item Case ForallRowL and ForallRowR. Symmetric to the ForallL and ForallR
    cases.
  \item Case \(\subtypes \declCtx {\rcd{R_1} {\rcd{R_2}}}\). Follows from the
    row case and DeclRecordWF.
  \end{itemize}
\end{proof}

\subsection{Substitution}

\begin{lem}[Substitution]
  If \(\wf \declCtx \tau\) and \(\subtypes {\declCtx, \alpha, \declCtx'} A B\),
  then \(\subtypes {\declCtx, [\alpha \mapsto \tau]\declCtx'} {[\alpha \mapsto
    \tau] A} {[\alpha \mapsto \tau]B}\).
\end{lem}

\begin{proof}
  The proof is by induction on the derivation.
  \begin{itemize}
  \item Case \(\subtypes \declCtx {\ell : A, R_1} {\ell : B, R_2}\).

    By the inductive hypothesis, \(\subtypes {\declCtx, [\alpha \mapsto
      \tau]\declCtx'} {[\alpha \mapsto \tau] R_1} {[\alpha \mapsto \tau]R_2}\)
    and \(\subtypes {\declCtx, [\alpha \mapsto \tau]\declCtx'} {[\alpha \mapsto
      \tau] A} {[\alpha \mapsto \tau]B}\). By \(\le\) Row, we conclude that
    \[\subtypes {\declCtx,[\alpha \mapsto \tau]\declCtx' } {\ell : [\alpha
        \mapsto \tau]A, [\alpha \mapsto \tau]R_1}{\ell : [\alpha \mapsto \tau]B,
      [\alpha \mapsto \tau]R_2}\]
  which, by definition of substitution, gives the desired result.
  \end{itemize}
\end{proof}

\subsection{Transitivity}
\begin{lem}[Transitivity of Declarative Subtyping]
  If \(\subtypes \declCtx A B\) and \(\subtypes \declCtx B C\) then \(\subtypes
  \declCtx A C\).
\end{lem}

\begin{proof}
  We induct on a similar metric used in the proof in \textit{Complete and Easy}.
  \[\langle \#\forall(B), \quad \#\rowall(B), \quad \mathcal{D}_1 + \mathcal{D}_2 \rangle\]
  where \(\#\forall\) is the count of quantifiers and \(\#\rowall\) is the count
  of row quantifiers. The last part is the size of each of the respective
  derviations in the hypothesis.

  \todo{TODO: reconcile the new metric and add in the rest of the cases that are
    affected by it.}
  \begin{itemize}
  \item Case \(\subtypes \declCtx {\ell : A, R_1} {\ell : B, R_2}\) and
    \(\subtypes \declCtx {\ell : B, R_2} {\ell : C, R_3}\). By the inductive
    hypothesis, \(\subtypes \declCtx R_1 R_2\) and \(\subtypes \declCtx R_2
    R_3\) implies \(\subtypes \declCtx R_1 R_3\). Similarly, we can conclude
    that \(\subtypes \declCtx A C\). Using these judgements, we can apply
    DeclRowWF to conclude the desired.
  \end{itemize}
\end{proof}

\subsection{Invertibility of \(\le \forall\)R}
\begin{lem}[Invertibility]
  If \(\mathcal{D}\) derives \(\subtypes \declCtx A {\forall \beta. B}\), then
  \(\mathcal{D}'\) derives \(\subtypes {\declCtx, \beta} A B\) where \(\mathcal
  D ' < \mathcal D\).
\end{lem}

\section{Algorithmic System Proof Extensions}

\subsection{Context Extension}

\subsection{Syntactic Properties}

Most (all?) of these proofs remain unchanged due to the fact that we haven't
done much to significantly alter the framework (although we have added a new
kind of evar, it functions much the same way).

\subsection{Instantiation Extends}

\begin{manuallemma}{32}[Instantiation Extends]
  If \(\instL \Gamma \eva \tau \Delta\) or \(\instR \Gamma \tau \eva \Delta\),
  then \(\Gamma \extends \Delta\).
\end{manuallemma}

\begin{proof}
  \begin{itemize}
  \item Case InstLRcd and InstRRcd: We'll employ a similar approach to the one
    used in Case InstLArr. We want to derive \(\Gamma_0[\eva] \extends \Delta\).

    Our IH gives us, by a lot of transitivity (which requires an inductive
    proof), that
      \[\Gamma_0[(\eva[k+1]),\eva[k],\dots,\eva[1],\eva = \{\ell_1 : \eva[1],
        \ell_2 : \eva[2], \dots, \ell_k : \eva[k], (\eva[k+1])\}] \extends \Delta\]

      We can employ a series of Lemma 28 (Unsolved Variable Addition for
      Extension), one Lemma 26 (Solution Admissibility for Extension), and
      transitivity to obtain
    \[\Gamma_0 \extends \Gamma_0[(\eva[k+1]),\eva[k],\dots,\eva[1],\eva = \{\ell_1 : \eva[1],
      \ell_2 : \eva[2], \dots, \ell_k : \eva[k], (\eva[k+1])\}]\],
    which, by the above and transitivity gives the desired.
  \end{itemize}
\end{proof}

\subsection{Subtyping Extends}

\begin{manuallemma}{33}[Subtyping Extension]
  If \(\subsumes \Gamma A B \Delta\), then \(\Gamma \extends \Delta\).
\end{manuallemma}
\begin{proof}
  By induction on the derivation.
  \begin{itemize}
  \item Case Record: Follows immediately from the IH
  \item Case Row: Follows from IH and transitivity. \(\Gamma \extends \Theta\)
    and \(\Theta \extends Delta\) implies the desired.
  \item Case Row Nil: Follows immediately from \(\Gamma\) being unchanged.
  \item Case RowMissingL and RowMissingR: Follows from instantiation extending.
  \item Case RowMissingLR: Follows from transitivity.
  \item Case RowTailReach: Follows from Lemma 28 (Solution Admissibility for
    Extension).
  \end{itemize}
\end{proof}

\subsection{Decidability of Instantiation}
\todo{fill in more specific details}
\begin{manuallemma}{34}[Left Unsolvedness Preservation]
  If \(\instL {\underbrace{\Gamma_0, \eva, \Gamma_1}_{\Gamma}} \eva A \Delta\) or
  \(\instR {\underbrace{\Gamma_0, \eva, \Gamma_1}_{\Gamma}} A \eva \Delta\), and
  \(\evb \in \mathsf{unsolved}(\Gamma_0), then \evb \in
  \mathsf{unsolved}(\Delta)\).
\end{manuallemma}

\begin{proof}
  The proof for the cases InstLRcd and InstRRcd follow from the same technique
  used for InstLArr.
\end{proof}

\begin{manuallemma}{35}[Left Free Variable Preservation]
\end{manuallemma}

\begin{manuallemma}{36}[Instantiation Size Preservation]
\end{manuallemma}

The proofs for the above are also similar.

\begin{thm}[Decidability of Instantiation]
\end{thm}

Follows from the lemmas with a similar technique to InstLArr.

\section{\emph{Complete and Easy} Errata}

A collection of minor errata from \emph{Complete and Easy} found while writing
this document. We use the authors' notations in this section, if ever we differ.

\textbf{\(\le \forall\) R, Declarative}

The rule, as stated:
\[
  \deduct[\(\le \forall\)R]
  {\subtypes {\declCtx, \beta} A B}
  {\subtypes \declCtx A {\forall \beta. B}}
\]

The problem:
\[
  \begin{prooftree}
    \hypo{\beta \in \cdot, \beta}
    \infer1{\wf{\cdot, \beta}{\beta \le \beta}}
    \infer1{\wf{\cdot}{\beta \le \forall \beta. \beta}}
  \end{prooftree}
\]
By Lemma 4, this implies \(\wf{\cdot}{\beta}\) which is not true.

The fix:
\[
  \deduct[\(\le \forall\)R Fixed]
  {\wf \declCtx A \spc \subtypes {\declCtx, \beta} A B}
  {\subtypes \declCtx A {\forall \beta. B}}
\]

\textbf{Poposition 2, Substitution}

There is a missing \([A/\alpha]\) on the context (as done in later rules, e.g.
A4, Substitution). 

The fix:

Write as the conclusion
\[
  \wf {\declCtx, [A/\alpha]\declCtx'} {[A/\alpha]B}.
\]

\textbf{Proof of A5, Transitivity}

There is a spurious \(\wf \declCtx \tau\) in the \(\le \forall\) R case.

\textbf{Lemma 19, Extension Equality Preservation}

In case AddSolved, the line

\textit{We implicity assume that \(\Delta'\) is well-formed, so \(\ev \alpha \notin
\mathsf{dom}(\Delta')\).}

is not quite correct. The line should be

\textit{We implicity assume that \(\Delta', \ev \alpha = \tau\) is well-formed,
  so \(\ev \alpha \notin \mathsf{dom}(\Delta')\).}

\end{document}